\documentclass[12pt]{article}
\date{}
\usepackage{tabularx}
\usepackage{capt-of}
\usepackage{caption}
\usepackage[most]{tcolorbox}
\usepackage[table]{xcolor}
\usepackage[backend=biber,
style=apa, sorting=nyt, defernumbers=true]{biblatex}
\addbibresource{referencias.bib}
\usepackage{setspace}
\setlength\bibitemsep{1.2\baselineskip}
\DeclareFieldFormat{labelnumber}{[#1]}
\definecolor{lavanda}{rgb}{0.827, 0.827, 1}
\definecolor{crema}{RGB}{251, 221, 173}
\usepackage{graphicx}
\usepackage{amsfonts}
\usepackage{amsmath}
\usepackage[colorlinks=true, linkcolor=blue, urlcolor=blue, citecolor=blue]{hyperref}
\usepackage[margin=2cm]{geometry}
\usepackage[spanish]{babel}
\usepackage{anyfontsize}
\usepackage{float}
\renewcommand\normalsize{\fontsize{14}{16}\selectfont}
\setlength{\parindent}{0pt}       

\begin{document}

\begin{titlepage}
    % Logo arriba izquierda
    \raggedright
    \includegraphics[scale=0.1]{udec} % <<--- logo
    
    % Espacio
    \vspace{3cm}
    
    % Título centrado con líneas arriba y abajo
    \centering
    \rule{0.7\textwidth}{1pt} \\[0.5cm]
    {\Huge\bfseries \textsc{Laboratorio N°1} \par}
    \vspace{0.5cm}
    \rule{0.7\textwidth}{1pt}\par
    
    \vfill
    
    % Autores centrados
    {\Large \textsc{Maximiliano Concha \\ Ignacio Villagrán \\ Iván Barría} \par}
    \vspace{0.5cm}
    {\Large \textsc{Profesor/Ayudante: \\ Claudio Alonso Faúndez Araya \\ Nikola Alexander Fabián Salazar Varas}}    \\
    \vspace{1cm}
    {\large Universidad de Concepción \\ 22 de septiembre de 2025}
\end{titlepage}
\newpage
\section*{Introducción}
A lo largo de la historia de la termodinámica, el estudio de los gases y 
su comportamiento frente a distintas condiciones (como de presión, 
volumen y temperatura) constituye uno de los pilares fundamentales de 
esta rama de la física. Las leyes de los gases ideales nos dan las 
herramientas para comprender y poder predecir el comportamiento de los 
sistemas a estudiar -en particular sistemas gaseosos-.  En el presente 
laboratorio se busca comprobar mediante una simulación (disponible en la 
siguiente pagina web \url{https://phet.colorado.edu/sims/html/gas-properties/latest/gas-properties_es.html}) 
dichas leyes, que explican de manera muy precisa como se comportan los 
gases bajo ciertas condiciones iniciales. \\

Sirviéndonos de la simulación, se entregaran gráficos que evidencien las 
relaciones entre las propiedades termodinámicas anteriormente mencionadas.
De este modo, mediante los datos obtenidos, se tiene como expectativa 
observar la validez de las leyes vistas a lo largo del curso como las 
leyes de Boyle, Charles y Gay-Lussac, y como consecuencia observar la 
validez de la ecuación general de los gases ideales $PV=nRT$. El estudio 
y análisis realizado permitirá  acrecentar la comprensión de los 
principios teóricos de manera experimental (gracias a los datos obtenidos 
de la simulación) y visual (entregando gráficos), a su vez acrecentando 
también el entendimiento de los principios matemáticos que rigen a todos 
estos procesos termodinámicos.

\section*{Marco Teórico}
Las leyes de los gases ideales son una herramienta muy importante a la 
hora de estudiar y analizar el comportamiento de gases sometidos a ciertas 
condiciones iniciales. \\

\textbf{Ley de los gases ideales:} La presión $P$, la temperatura $T$, y el volumen 
$V$ de un gas ideal, están relacionados por una simple fórmula llamada 
la ley del gas ideal, cuya expresión matemática es:
\[
PV=nRT
\]
Donde $P$ es la presión del gas, $V$ es el volumen que ocupa, $T$ es su 
temperatura, $R$ es la constante del gas ideal, y $n$ es el número de moles
del gas. \parencite{khanleygasesideales} \\

Las siguientes leyes son casos particulares de la ley de los gases ideales,
donde se mantiene una variable constante. $T_1, V_1, P_1$ son las condiciones
iniciales y $T_2, V_2, P_2$ las condiciones finales de temperatura, 
volumen y presión respectivamente.
\newpage
\textbf{Ley de Boyle:} Establece que la presión de un gas en un recipiente 
cerrado es inversamente proporcional al volumen del recipiente, cuando la 
temperatura es constante. \parencite{educaplusleyboyle} Matemáticamente se 
expresa como:
\[
P_1V_1=P_2V_2
\]

\textbf{Ley de Charles:} Establece que, a presión 
constante, el volumen de un gas es directamente proporcional a su 
temperatura absoluta. \parencite{masamleycharles} Matemáticamente, esta 
relación se expresa de la siguiente manera:
\[
\dfrac{V_1}{T_1}=\dfrac{V_2}{T_2}
\]

\textbf{Ley de Gay-Lussac:} Cuando aumenta la temperatura de una muestra 
de gas en un recipiente rígido, también aumenta la presión del gas. El 
aumento de la energía cinética hace que las moléculas de gas golpeen las 
paredes del recipiente con mayor fuerza, lo que genera una mayor presión.
\parencite{libretextsgaylussac} Su expresión matemática es:
\[
\dfrac{P_1}{T_1}=\dfrac{P_2}{T_2}
\]

\section*{Materiales}
\begin{itemize}
    \item Recipiente con gas
    \item Pistón
    \item Termómetro
    \item Barómetro
    \item Regularizador de temperatura
    \item Bomba de moléculas
\end{itemize}

\section*{Procedimientos}
nose papu :V
\section*{Conclusión}


\newpage
\printbibliography[heading=bibintoc]

\end{document}